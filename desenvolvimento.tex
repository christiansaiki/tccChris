%!TEX root = index.tex
\chapter{Testes e Resultados}
\label{cha:testes_e_resultados}
Conforme apresentado anteriormente, o produto da empresa Beeconnect é o aplicativo com o mesmo o nome. Tal produto foi desenvolvido para as plataformas Android e iOS e consiste basicamente em um aplicativo para descontos em lojas físicas. O seu principal diferencial é a geolocalização indoor precisa com uso de um aparelho chamado beacon. Quando um usuário do aplicativo passar por um beacon localizado dentro de uma loja parceira ele pode receber uma notificação informando que ele recebeu um desconto especial em um produto relevante ou receber um simples \enquote{Bem vindo} conforme mostrado na \autoref{fig:bemvindo_beeconnect}.

\begin{figure}[H]
\caption{Exemplo de notificação do aplicativo Beeconnect}
\centerline{\includegraphics[scale=0.35]{img/bemvindo_beeconnect}}
\label{fig:bemvindo_beeconnect}
\caption* {Fonte: https://app.beeconnect.com.br/}
\end{figure}

Até o momento em que os testes foram realizados a Beeconnect contava com cerca de 2000 downloads do aplicativo, 1600 usuários cadastrados e 10 empresas parceiras. Um dos problemas é que nenhuma loja ainda estava disposta a pagar pela plataforma.

Os principais desafios da Beeconnect são:
\begin{itemize}
\item Crescer a base de lojas pagantes do aplicativo.
\item Crescer a base de usuários de forma barata.
\end{itemize}

Como mencionado previamente tais desafios eram realmente difíceis de serem resolvidos porque muitos usuários só baixariam o aplicativo se ele possuísse mais lojas participantes, assim como muitas lojas só se interessavam pela base de usuários e só entrariam no aplicativo caso a base fosse grande, com mais de cem mil usuários.

Baseando-se na metodologia apresentada no capítulo anterior o autor então começou a desenvolver os testes e resultados que serão apresentados nesse capítulo. Foram feitas duas iterações no ciclo explicitado pela metodologia introduzida no capítulo anteriror.

\section{Mapear estado atual da startup}
\label{cha:mapear_estado}
O autor elaborou um canvas inicial, ilustrado na \autoref{fig:canvas_beeconnect_1}, baseado nas premissas iniciais da empresa: 

\begin{figure}[H]
\caption{Canvas de Modelo de Negócio inicial da Beeconnect}
\centerline{\includegraphics[scale=0.25]{img/canvas_beeconnect_1}}
\label{fig:canvas_beeconnect_1}
\caption* {Fonte: Elaborado pelo autor}
\end{figure}

Foram detalhados então cada um dos nove blocos do Canvas de Modelo de Negócio.
\subsection{Segmentos de Clientes}
\label{cha:segmentos_de_clientes}
Os segmentos de clientes para o aplicativo são:
\begin{itemize}
\item Varejistas de lojas físicas, que serão tratados daqui em diante como \enquote{Lojas} ou \enquote{Varejistas}: 
\item Frequentadores de lojas físicas que gostam de descontos, que serão tratados aqui em diante como \enquote{Usuários}: 
\end{itemize}
Deste modo verificou-se que o produto atende dois segmentos distintos porém dependentes, típico de um mercado Multilateral. Por exemplo: sem uma boa diversidade de Lojas participantes no aplicativo não há muitas opções de descontos para os Usuários.

\subsection{Proposição de Valor}
\label{cha:proposicao_de_valor}
Dado que o produto atende um segmento multilateral de clientes ele tem que gerar valor para ambos os segmentos.
\begin{itemize}
\item Novo canal de relacionamento com o cliente: os lojas físicas tem no aplicativo uma nova plataforma para se comunicar com seus clientes. Elas podem enviar notificações para eles quando ele passar em um raio de quinhentos metros de uma de suas lojas, graças a tecnologia do GPS. Além disso, o cliente pode consultar as promoções de uma determinada loja sem precisar sair de casa.
\item Fidelização do cliente: graças a tecnologia de beacon o lojista consegue saber quantas vezes cada cliente foi a sua loja a premiá-lo de acordo com isso seja com descontos ou com algum brinde.
\item Descontos Exclusivos: é a principal proposta de valor para os Usuários. Mais uma vez graças a tecnologia de beacons o aplicativo consegue saber quando que o usuário está próximo de um determinado produto e oferecer um desconto exclusivo.
\item Fácil acesso a informações das lojas físicas: o usuário pode facilmente consultar onde fica a loja de supermercado mais próxima a ele, ver seu endereço e já rapidamente colocar no endereço no GPS.
\end{itemize}

\subsection{Relacionamento com Cliente}
\label{cha:relacionamento_com_cliente}
Dado que as Lojas serão os clientes pagantes e o número é bem menor que o de usuários do aplicativo a empresa optou por oferecer uma comunicação personalizada com os varejistas e uma comunicação automatizada com os usuários. 

\subsection{Canais}
\label{cha:canais}
Os principais canais são:
\begin{itemize}
\item Website: com o \textit{site} https://app.beeconnect.com.br/ é possível atender tanto o usuário quanto o lojista. Nele o lojista pode fazer fazer o gerenciamento das campanhas que irão aparecer no aplicativo. Já o usuário pode saber mais sobre o aplicativo.
\item Facebook: a página do aplicativo no Facebook, https://www.facebook.com/beeconnectbr/, foi feita com o intuito de conseguir fazer campanhas para conseguir mais usuários para o aplicativo, mas também é um canal de interação com as Lojas dado que é possível, por exemplo, mencionar a loja em um publicação da página da Beeconnect.
\item Relações Públicas: dado que a BC faz parte do grupo TM que possui uma assessoria de relações públicas há chances da BC aparecer em reportagens.
\item Lojas de Aplicativos: as lojas de aplicativos \textit{App Store} do sistema operacional iOS e \textit{Play Store} do sistema operacional Android são de extrema importância pois são nelas que o usuário consegue baixar o aplicativo para o celular. 
\item Email: Através do email marketing é possível se relacionar com os usuários já existentes para informá-los sobre novas lojas parceiras ou sobre descontos super especiais.
\end{itemize}

\subsection{Fluxos de Receita}
\label{cha:fluxos_de_receita_canvas_bc_1}
A empresa optou por oferecer dois métodos de cobrança dos lojistas:
\begin{itemize}
\item Assinatura de R\$99 por mês por beacon por loja: assim se uma loja optar por utilizar 2 beacons ela terá que pagar R\$198 por mês. Nessa assinatura o lojista ganha acesso a todas as opções como enviar uma notificação assim que o cliente entra na loja, acesso ao número de visitantes que passaram pela loja, entre outras funcionalidades.
\item Assinatura de R\$49 por mês por loja: a empresa optou por oferecer essa modalidade de assinatura para o caso do lojista não ver valor no uso dos beacons. Assim ele só conta com a funcionalidade de enviar notificações para os usuários estiverem a um raio de quinhentos metros de sua loja e disponibilizar seus produtos na vitrine virtual do aplicativo.
\end{itemize}

\subsection{Parcerias Chave}
\label{cha:parcerias_chave}
As parcerias-chave da Beeconnect são:
\begin{itemize}
\item Amazon Web Services: A Amazon Web Services, popularmente conhecida como AWS é o serviço de computação em nuvem da Amazon, maior site de compras \textit{online} dos Estados Unidos. A AWS é de fundamental importância para um negócio que envolve servidores, gracas a ela muitos negócios se tornam viáveis por é possível testar hipóteses sem gastar muito dinheiro para construir toda uma infraestrutura de servidores por trás. Com a AWS o empreendedor só paga por hora de máquina utilizada e dá para facilmente colocar uma máquina melhor caso a infraestrutura necessite para suportar um tráfego maior.
\item Apple e Google: Um desenvolvedor de aplicativos pode se manter fora das lojas de aplicativos da Apple e do Google, entretanto se ele quiser ser levado a sério ele tem que passar por todo o trâmite de aprovação de seu aplicativo para poder disponibilizá-lo nas lojas oficiais. 
\item Vendedor terceirizado: Um vendedor entrou em contato com a equipe pois ele acabou sabendo do produto e achou interessante. Ele acabou propondo vender o produto mediante a uma comissão de 20\% por venda. Dado que o time de vendas da BC é bem enxuto a equipe achou interessante a proposta dado que só haveria um custo variável por venda realizada.
\item TM: Poucas empresas tem a chance de serem criadas dentro de um grupo que já possui startups lucrativas. A TM forneceu uma estrutura muito boa para o desenvolvimento da Beeconnect.
\end{itemize}

\subsection{Atividades Chave}
\label{cha:parcerias_chave}
As atividades Chave da Beeconnect são:
\begin{itemize}
\item Manutenção e constante melhoria do aplicativo: os desenvolvedores devem sempre estar atentos à mudanças nos sistemas operacionais. Por exemplo: em 2015 com o lançamento da versão 9 do sistema da Apple alguns códigos tiveram que ser alterados caso contrário o aplicativo não iria funcionar, o mesmo aconteceu para a versão \textit{Marshmallow} do sistema operacional da Google. Além disso, os desenvolvedores necessitam colocar mais funcionalidade ao aplicativo além de possibilitar a realização de testes A/B na interface para que ela seja a mais intuitiva possível.
\item Divulgação do Aplicativo: O desafio conforme explicado por \citeonline{mcclure2007startup} é conseguir o meio mais barato de adquirir bastantes usuários. O custo de aquisição de usuários deve ser menor que a receita gerada por cada usuário.
\item Venda do serviço para varejistas: Assim como a base de usuários tem que crescer, a base de lojas também deve crescer junto. Por se tratar de um negócio multilateral quanto mais lojas melhor para os usuários, assim como quanto mais usuários melhor é para as lojas.
\end{itemize}

\subsection{Recursos Chave}
\label{cha:recursos_chave}
Os Recursos Chave da Beeconnect são:
\begin{itemize}
\item Time: A equipe é bem qualificada, praticamente toda formada por engenheiros e estudantes de engenharia da Escola Politécnica da USP.
\item Beacons: Os beacons são equipamentos pouco conhecidos no mercado brasileiro, entretanto, já estão sendo utilizados amplamente nos Estados Unidos. Esses aparelhos são relativamente baratos se comparados com outras ferramentas de localização interna. 
\item Computação em nuvem: Conforme citado anteriormente a computação em nuvem permite que a empresa possa testar suas hipóteses e criar seus negócios sem que haja um investimento adiantado em servidores. Nesses servidores ficam os códigos responsáveis pela comunicação com o aplicativo e pela interação do usuário com o site da Beeconnect.
\item Programa para comunicação com beacons: Os desenvolvedores tiveram que fazer um programa que possibilita a comunicação com beacons. Tal \textit{software} possibilita a comunicação entre o celular do usuário com o beacon, além disso, ele já envia para os servidores da Beeconnect qual beacon que o celular está captando, assim o servidor pode mandar uma promoção especial para o usuário que está naquela loja. Esse \textit{software} pode ser instalado em outras aplicações que queiram se comunicar com os beacons da Beeconnect.
\end{itemize}

\subsection{Estrutura de Custo}
\label{cha:estrutura_de_custo}
A Estrutura de Custo da Beeconnect é descrita abaixo:
\begin{itemize}
\item Time: A equipe é responsável pela maior parte dos custos da empresa. Com cerca de 12 membros no time, a Beeconnect gasta quase R\$100.000 em recursos humanos.
\item Rateio do aluguel e despesas do escritório: A TM possui um escritório localizado na Rua Haddock Lobo. O aluguel e demais despesas do escritório são rateados proporcionalmente ao número de integrantes por empresa da TM.
\item Beacons: Os aparelhos são importados da China. Cada aparelho sai por cerca de R\$100 já com impostos e frete.
\item Marketing e Vendas: Nesse item podem ser considerados as custos das campanhas de marketing digital e físico bem como os gastos para visitar clientes.
\item Mongo Lab: É o serviço de base de dados utilizado para guardar os dados dos usuários, campanhas, lojas. Gasta-se cerca de R\$600 por mês com esse serviço para armazenar até 40 Gb
\item Redis Labs: É um outro serviço de base de dados, esse tipo de base é muito mais rápido pois ele um tipo de memória de acesso mais rápido, entretanto o custo de armazenamento é mais caro. Gasta-se cerca de R\$60 por mês para o armazenamento de até 0.5 Gb
\item Github: é um serviço de armazenamento, versionamento e compartilhamento de código.
\item AWS: Como dito anteriormente, é o serviço de computação em nuvem da Amazon. A Beeconnect utiliza cerca de 20 máquinas e gasta por volta de R\$1600 por mês.
\end{itemize}

\section{Gerar hipóteses sobre a proposta de valor da empresa}
\label{cha:gerar_hipoteses}
Dada a situação inicial da empresa modelada no Canvas de Modelo de Negócio e o objetivo do trabalho de formatura de provar que a Beeconnect tem um modelo de negócio sustentável, o autor recorreu a literatura para buscar a melhor alternativa de solução para o problema. 
Segundo a literatura o principal problema de uma startup é construir um produto que ninguém quer. Em outras palavras, o maior problema é o produto construído ou serviço prestado não gerar valor para seu cliente.
Ficou evidente que era urgente verificar se o aplicativo gerava valor para seus usuários. Baseando-se nas métricas pirata de \citeonline{mcclure2007startup}, na análise de coorte de \citeonline{leanstartup} e no Canvas de Modelo de Negócio, nos itens Segmentos de Clientes e Proposição da Valor, foram elaboradas as seguintes hipóteses:
\begin{itemize}
\item Hipótese 1: \textbf{Varejistas de lojas físicas tem interesse num novo canal de relacionamento com os clientes}. O objetivo era verificar se os varejistas estavam interessados em se comunicar com seus clientes através de notificações em \textit{smartphone} além dos canais tradicionais como email, telefone, panfletos entre outros.
\item Hipótese 2: \textbf{Varejistas de lojas físicas tem interesse em fidelizar o cliente}. O objetivo era testar se os varejistas estavam dispostos a dar algum benefício aos seus clientes antigos para que eles voltassem cada vez mais às suas lojas.
\item Hipótese 3: \textbf{Post no \textit{Facebook} focando os simpatizantes de uma determinada marca é um meio barato e efetivo de adquirir clientes}. O teste de tal hipótese visava verificar se o \textit{Facebook} seria um bom meio de conseguir novos usuários para o aplicativo dado que o crescimento rápido da base de usuários de um modo barato era de extrema importância para o sucesso do negócio. Já se sabia de antemão que o marketing nessa rede social era um pouco caro, entretanto, a dúvida que restava era se o foco em determinado público ajudaria a aumentar a conversão de modo a reduzir o custo um novo usuário.
\item Hipótese 4: \textbf{É fácil de gerar cupons na primeira semana de uso do aplicativo}. Tal hipótese visava testar a questão da Ativação do usuário apontada por \citeonline{mcclure2007startup}. Infelizmente, devido a falta de integração com os sistemas de vendas dos varejistas era impossível a verificação automática se o usuário havia de fato \textbf{utilizado} do cupom de desconto. Portanto, para a equipe da Beeconnect o usuário só era considerado ativo se ele tivesse \textbf{gerado} pelo menos um cupom. 
\item Hipótese 5: \textbf{Os usuários que geraram cupom utilizam o aplicativo na segunda semana}. O intuito de verificar essa hipótese baseia-se no conceito de Retenção do usuário mencionado tanto por \citeonline{mcclure2007startup} como por \citeonline{leanstartup}. O usuário só seria considerado retido se ele estivesse gerando cupons constantemente no aplicativo.
\item Hipótese 6: \textbf{Clientes que gostam de desconto tem interesse em fácil acesso às informações das lojas físicas}. Dado o contexto que os usuários do aplicativo são frequentadores de lojas físicas a equipe da Beeconnect achou que seria útil construir uma página com informações sobre a loja como localização com integração com o sistema do \textit{Google Maps} e inclusive com um que levava ao GPS do \textit{smartphone}. Tais funcionalidades foram uma das que mais demandaram tempo do time de desenvolvimento sendo que não houve nenhuma demanda clara por parte dos usuários. O objetivo seria testar se os usuários realmente utilizariam essa funcionalidade ou se ela foi feita em vão. 
\end{itemize}

\section{Desenhar Testes de Hipóteses}
\label{cha:desenhar_hipoteses}
Uma vez que as hipóteses foram elaboradas foi necessário estruturar a maneira como tais hipóteses seriam testadas. 

\subsection{Hipótese 1}
\label{cha:hip_1}
Para a hipótese \enquote{Varejistas de lojas físicas tem interesse num novo canal de relacionamento com os clientes} o teste proposto foi organizar um mutirão de vendas e sair para a rua, na região dos Jardins devido a proximidade com o escritório da Beeconnect e tentar vender o serviço. A métrica de sucesso definida foi:
\[ \dfrac{NLP}{LV} > 10\%\]

Onde: 
\begin{itemize}
\item NLP: é o número de novas lojas parceiras que foram adquiridas durante o mutirão de vendas.
\item LV: é o número de lojas visitadas durante o mutirão de vendas.
\end{itemize}

\subsection{Hipótese 2}
\label{cha:hip_2}
Para a hipótese \enquote{Varejistas de lojas físicas tem interesse em fidelizar o cliente} os vendedores da Beeconnect deveriam ligar para os varejistas que já estavam participando no aplicativo e propor que eles oferecessem pelo um produto com desconto exclusivo para o aplicativo de formar a fidelizar os clientes. Nesse caso a métrica de sucesso definida foi:

\[\dfrac{LCD}{LP} > 40\%\]

Onde: 
\begin{itemize}
\item LCD: é o número de lojas que disponibilizaram pelo menos um cupom de desconto exclusivo.
\item LP: é o número de lojas participantes no aplicativo, ou seja, lojas que já estavam com pelo menos alguma campanha disponível no app.
\end{itemize}

Para as demais hipóteses o teste proposto foi analisar um certo grupo de usuários e verificar como eles se comportariam com o passar do tempo. 

\subsection{Hipótese 3}
\label{cha:hip_3}
O primeiro passo consistiria na criação de um anúncio pago no Facebook colocando como alvo as pessoas que deram \enquote{curtir} na página em um dos parceiros do aplicativo na época. A métrica utilizada para testar a hipótese \enquote{Post no Facebook para simpatizantes da marca é um meio barato e efetivo de adquirir clientes} foi:
\begin{align*} 
CAC < R\$1,00 \\
\dfrac{D}{AP} > 0.5\%
\end{align*}

Onde: 
\begin{itemize}
\item CAC: é o custo de aquisição de um usuário. É o custo de marketing para conseguir um novo usuário do aplicativo. Um dos grandes objetivos das startups é ter um CAC menor que o valor do ciclo de vida do cliente que basicamente é todo o valor gerado pelo cliente durante o uso do aplicativo.
\item D: é o número de downloads do aplicativo advindos do post no \textit{Facebook}.
\item AP: é o alcance da publicação, ou seja, quantas pessoas diferentes viram a publicação feita no Facebook. O grande desafio é fazer com que as pessoas interajam com a publicação paga, pois a cada comentário, curtida e compartilhamento a publicação ganha um impulso maior, ou seja, ela é exibida para mais pessoas a um custo menor. Tudo isso ocorre porque o algoritmo do Facebook interpreta que o post possui um conteúdo relevante e interessante para seus usuários, algo que transcende o simples marketing.
\end{itemize}

\subsection{Hipótese 4}
\label{cha:hip_4}
O segundo passo seria analisar como cada uma dessas pessoas que baixaram o aplicativo através desse post iriam se comportar dentro do aplicativo. Durante a primeira semana de uso para mensurar o sucesso da hipótese \enquote{É fácil de gerar cupons na primeira semana de uso do aplicativo}, a equipe decidiu medir a métrica:

\[\dfrac{UP}{D} > 10\%\]

Onde: 
\begin{itemize}
\item UP: é o número de usuários que geraram cupons durante a primeira semana de uso do aplicativo.
\item D: é o número de downloads do aplicativo advindos do post no \textit{Facebook}.
\end{itemize}

\subsection{Hipótese 5}
\label{cha:hip_5}
Para a hipótese \enquote{Os usuários que geraram cupom utilizam o aplicativo na segunda semana} a métrica utilizada foi:
\[\dfrac{UPS}{UP} > 25\%\]

Onde: 
\begin{itemize}
\item UPS: é o número de usuários que geraram cupons durante a primeira e a segunda semana de uso do aplicativo.
\item UP: é o número de usuários que geraram cupons durante a primeira semana de uso do aplicativo.
\end{itemize}

\subsection{Hipótese 6}
\label{cha:hip_6}
E para testar a hipótese \enquote{Clientes que gostam de desconto tem interesse em fácil acesso às informações das lojas físicas} seria necessário utilizar a ferramenta \textit{Google Analytics} para checar quantas pessoas dentro desse grupo de usuários consultaram a página de informação da loja física no aplicativo, ilustrada na \autoref{fig:info_parceiros}. Portanto a métrica utilizada para checar tal hipótese foi:
\[\dfrac{UPI}{D} > 30\%\]

Onde: 
\begin{itemize}
\item UPI: é o número de usuários que consultaram a página de informação da Sonheria Dulca.
\item D: é o número de downloads do aplicativo advindos do post no \textit{Facebook}.
\end{itemize}

\begin{figure}[H]
\caption{Tela de Informações do Parceiro}
\centerline{\includegraphics[scale=0.13]{img/info_parceiros}}
\label{fig:info_parceiros}
\caption* {Fonte: Aplicativo Beeconnect}
\end{figure}

\section{Testar Hipóteses}
\label{cha:testar_hipoteses}
Durante o mês de março a equipe focou em realizar os testes o mais rápido possível. 

\subsection{Teste da hipótese 1}
\label{cha:teste_1}
Para testar a hipótese 1 foi organizado um mutirão de vendas durante uma segunda-feira que focou na região da Paulista entre os metrôs Trianon-Masp e Consolação devido a proximidade com o escritório. Onze membros da Beeconnect participaram do mutirão organizados em cinco grupos, quatro duplas e um trio, cada grupo deveria percorrer uma das áreas, ilustradas na \autoref{fig:area_mutirao}.

\begin{figure}[H]
\caption{Área coberta pelo mutirão de vendas}
\centerline{\includegraphics[scale=0.25]{img/area_mutirao}}
\label{fig:area_mutirao}
\caption* {Fonte: Elaborado pelo autor através da ferramenta do Google Maps}
\end{figure}

Durante o mutirão de vendas os membros da equipe tentaram vender primeiro o plano mensal com beacons que custava R\$99 por mês por beacon por loja, caso não houvesse interesse o membro oferecia o plano sem beacon que custava R\$49 por mês por loja. A equipe também carregava uma apresentação impressa, exemplo de slide na \autoref{fig:apresentacao_vendas_bc}, da Beeconnect para deixar com os funcionários da loja caso o dono não estivesse presente.

\begin{figure}[H]
\caption{Exemplo de slide da Apresentação de Vendas}
\centerline{\includegraphics[width=0.8\textwidth]{img/apresentacao_vendas_bc}}
\label{fig:apresentacao_vendas_bc}
\caption* {Fonte: Material de vendas da Beeconnect}
\end{figure}

\subsection{Teste da hipótese 2}
\label{cha:teste_2}
Para o teste da hipótese 2, que foi realizado dois dias após o mutirão de vendas, o time comercial da startup ligou para todos os varejistas que estavam participando no aplicativo para perguntar se eles estavam dispostos a dar pelo menos um desconto exclusivo no aplicativo.

\subsection{Teste da hipótese 3}
\label{cha:teste_3}
O teste da hipótese 3 foi realizado dois dias após a realização do teste 2, ou seja, em uma sexta-feira. A equipe da Beeconnect criou um post pago no Facebook, conforme mostra a \autoref{fig:post_sonheria}, com a melhor oferta disponível no aplicativo na época, \enquote{50\% de desconto na compra do segundo sonho} na Sonheria Dulca. Na campanha de marketing a equipe mirou no público que havia dado curtir na página do Facebook da Sonheria. A campanha rodou por 1 dia por motivos de orçamento.

\begin{figure}[H]
\caption{Post Facebook Sonheria Dulca}
\centerline{\includegraphics[width=0.8\textwidth]{img/post_sonheria}}
\label{fig:post_sonheria}
\caption* {Fonte: Material de marketing da Beeconnect}
\end{figure}

\subsection{Teste da hipótese 4}
\label{cha:teste_4}
Para o teste da hipótese 4, que começou um dia após a realização do teste 3, utilizou-se do monitoramento desenvolvido dentro do aplicativo para analisar o comportamento dos usuários que baixaram o app através do post feito no Facebook. O objetivo era ver quantos usuários únicos iriam gerar cupom durante a primeira semana de experiência do aplicativo.

\subsection{Teste da hipótese 5}
\label{cha:teste_5}
Para testar a hipótese 5 foi realizado após uma semana em relação ao teste 4 com intuito de fazer uma análise de coorte do grupo de usuários que foi impactado pelo post feito no Facebook. Analisou-se quantos usuários que geraram cupom na semana anterior também geraram cupom naquela semana.

\subsection{Teste da hipótese 6}
\label{cha:teste_6}
O teste da hipótese 6 foi realizado em paralelo com os teste 4 e 5. Através do monitoramento feito no aplicativo foi possível analisar quantos usuários únicos acessaram a tela de informações da Sonheria Dulca, ilustrada na \autoref{fig:info_parceiros}.

\section{Analisar Resultados e Repetir Ciclo}
\label{cha:analisar_resultados}
Essa seção apresenta quais foram os resultados obtidos durante a fase de testes assim como faz uma análise desses resultados.

\subsection{Resultado do Teste da hipótese 1}
\label{cha:resultado_1}
Durante esse teste foram visitadas 70 lojas, sendo que 10 decidiram fazer parte do aplicativo, ou seja, pouco mais que 14\% das lojas foram convertidas, como mostra a \autoref{tab:resultado_1}. Dado que a porcentagem esperada de lojas convertidas era de 10\%, pode-se dizer que a Beeconnect passou no teste.
\begin{table}[H]
\centering
\caption{Resultado do teste da hipótese 1}
\label{tab:resultado_1}
\begin{tabular}{|c|c|c|c|}
\hline
Número de Lojas Visitadas & Número de Lojas Convertidas & Resultado & Esperado          \\ \hline
70                        & 10                          & 14\%      & \textgreater 10\% \\ \hline
\end{tabular}
\caption* {Fonte: Elaborado pelo autor}    
\end{table}

\subsection{Resultado do Teste da hipótese 2}
\label{cha:resultado_2}
Ao ligar para os varejistas, os vendedores da Beeconnect conseguiram que 12 das 28 lojas parceiras no momento disponibilizassem descontos exclusivos para o aplicativo, ou seja, pouco mais que 42\%, como mostra a \autoref{tab:resultado_2}. Nesse teste também pode-se concluir que a startup passou no teste dado que seu resultado de 42\% foi maior que o esperado de 40\%.
\begin{table}[H]
\centering
\caption{Resultado do teste da hipótese 2}
\label{tab:resultado_2}
\begin{tabular}{|c|c|c|c|}
\hline
Lojas com Desconto Exclusivo & Lojas Parceiras & Resultado   & Esperado          \\ \hline
12                       	  & 28              & 42.8\%      & \textgreater 40\% \\ \hline
\end{tabular}
\caption* {Fonte: Elaborado pelo autor}    
\end{table}

\subsection{Resultado do Teste da hipótese 3}
\label{cha:resultado_3}
O post no Facebook feito para a campanha da Sonheria rendeu 90 downloads do aplicativo, sendo visto por 55440 vezes, o que deu uma conversão de 0,23\%, que ficou bem abaixo do esperado de 1\%, referenciado na \autoref{tab:resultado_3_1}. Cada download custou cerca de R\$ 4,43, bem acima que o esperado de R\$1,00, como visto na \autoref{tab:resultado_3_2}. Portanto, os dois resultados do teste com o post no Facebook foram bem insatisfatórios para a Beeconnect.
\begin{table}[H]
\centering
\caption{Resultado 1 do teste da hipótese 3}
\label{tab:resultado_3_1}
\begin{tabular}{|c|c|c|c|}
\hline
Downloads & Alcance & Conversão & Conversão Esperada \\ \hline
90                       & 55440 &  0.23\%     & \textgreater 0.5\% \\ \hline
\end{tabular}
\caption* {Fonte: Elaborado pelo autor}    
\end{table}

\begin{table}[H]
\centering
\caption{Resultado 2 do teste da hipótese 3}
\label{tab:resultado_3_2}
\begin{tabular}{|c|c|c|}
\hline
Custo por Download & Custo por Download Esperado \\ \hline
R\$4,43          & \textless R\$1,00 \\ \hline
\end{tabular}
\caption* {Fonte: Elaborado pelo autor}    
\end{table}


\subsection{Resultado do Teste da hipótese 4}
\label{cha:resultado_4}
Dos 90 usuários que baixaram o aplicativo através do Post do Facebook criado para a hipótese 3, 11 conseguiram gerar ao menos um cupom na primeira semana de uso do aplicativo, uma conversão de pouco mais que 12\%. Portanto, o resultado foi positivo para empresa dado que o esperado era que 10\% dos usuários gerassem cupons, como mostra a \autoref{tab:resultado_4}.
\begin{table}[H]
\centering
\caption{Resultado do teste da hipótese 4}
\label{tab:resultado_4}
\begin{tabular}{|c|c|c|c|}
\hline
Usuários & Usuários Únicos com Cupom & Conversão & Conversão Esperada \\ \hline
90       & 11  & 12.2\%   & \textgreater 10\% \\ \hline
\end{tabular}
\caption* {Fonte: Elaborado pelo autor}    
\end{table}

\subsection{Resultado do Teste da hipótese 5}
\label{cha:resultado_5}
Dos 11 usuários que geraram pelo menos um cupom na primeira semana de uso do aplicativo, 3 geraram pelo menos um cupom na segunda semana do aplicativo, pouco mais de 27\%, conforme mostra a \autoref{tab:resultado_5}. Como mencionado anteriormente, o resultado esperado era de 25\%, portanto a empresa teve um resultado positivo quanto a um teste de retenção.
\begin{table}[H]
\centering
\caption{Resultado do teste da hipótese 5}
\label{tab:resultado_5}
\begin{tabular}{|c|c|c|c|}
\hline
Usuários cupom semana 1 & Usuários cupom semana 2 & Resultado & Esperado \\ \hline
11       & 3  & 27.2\%   & \textgreater 25\% \\ \hline
\end{tabular}
\caption* {Fonte: Elaborado pelo autor}    
\end{table}

\subsection{Resultado do Teste da hipótese 6}
\label{cha:resultado_6}
Quanto ao teste de consultas à tela de informações do parceiro, dos 90 usuários que baixaram o aplicativo, 15 consultaram ao menos uma vez essa tela durante as duas semanas de monitoramento, cujo resultado dá aproximadamente 17\%, muito abaixo dos 30\% esperados, como mostra a \autoref{tab:resultado_6}. Pode-se concluir então que a tela com a funcionalidade de informações do parceiro foi em vão e que talvez o tempo dos desenvolvedores poderia ser focado em outras funcionalidades mais importantes. 

\begin{table}[H]
\centering
\caption{Resultado do teste da hipótese 6}
\label{tab:resultado_6}
\begin{tabular}{|c|c|c|c|}
\hline
Usuários & Usuários que consultaram a tela de informações & Resultado & Esperado \\ \hline
90       & 15  & 16.7\%   & \textgreater 30\% \\ \hline
\end{tabular}
\caption* {Fonte: Elaborado pelo autor}    
\end{table}

\section{Listar Lições Aprendidas do Primeiro Ciclo}
\label{cha:listar_licoes_aprendidas}
Durante o mutirão de vendas a equipe notou que se tivesse levado uma câmera digital ou levado um cartão de memória para conseguir as fotos dos produtos de uma loja seria muito mais rápido de integrar com novos varejistas.
O fato de levar os beacons para apresentá-los no mutirão foi um fator relevante para mostrar o diferencial do produto. As equipes que não levaram o aparelho tiveram um desempenho pior que as demais.
Para que os varejistas entendessem melhor sobre o que aplicativo estava tentando oferecer os membros utilizaram termos como \enquote{o aplicativo é como um jornal do bairro no celular}, \enquote{o aplicativo é uma extensão da vitrine física no celular}, \enquote{trabalhamos com marketing digital}. Tal estratégia pareceu funcionar bem.

Todos as lojas que estavam no aplicativo não estavam pagando pelo uso do serviço. Uma maneira de alavancar a parceria foi pedir para que as lojas postassem em suas páginas do Facebook dizendo que elas estavam fazendo parte da Beeconnect, conforme ilustrado na \autoref{fig:paulista_burger}.

\begin{figure}[H]
\caption{Post da Página do Paulista Burger}
\centerline{\includegraphics[width=0.8\textwidth]{img/paulista_burger}}
\label{fig:paulista_burger}
\caption* {Fonte: https://www.facebook.com/paulistaburger/}
\end{figure}

A publicação paga no \textit{Facebook} tendo como público-alvo uma determinada página não pareceu funcionar tão bem, a conversão foi relativamente alta se comparar com a média de mercado. Entretanto, o alto custo por download deixa inviável tal iniciativa.

Outro aprendizado é de tentar validar se há uma demanda real por uma determinada funcionalidade no aplicativo, como aconteceu com o item da hipótese 6, a tela de informações do parceiro. Dado que o time de desenvolvimento tem um tempo escasso e valioso o foco deve ser alocar os recursos do time só em funcionalidades que claramente geram valor para seus usuários.

Os resultados quanto a retenção dos clientes foi melhor que o esperado, entretanto deve-se medir ainda mais a retenção no decorrer de meses. Dados os resultados dos testes o autor decidiu iterar pelo ciclo mais uma vez para aprender mais sobre o modelo de negócio da startup, principalmente na questão se os clientes estariam dispostos a pagar pelo serviço prestado.

\section{Mapear estado atual da startup 2}
\label{cha:mapear_estado_2}
A partir da evolução na compreensão do modelo de negócio da Beeconnect a equipe decidiu elaborar um novo Canvas de Modelo Negócio, referenciado na \autoref{fig:canvas_beeconnect_2}, para que ficasse mais fácil de visualizar os novos aprendizados. Só foram feitos alguns ajustes em relação ao canvas inicial nos blocos:

\begin{itemize}
\item Proposta de Valor: houve um ajuste na formulação dos descontos exclusivos para \enquote{Cupons de descontos exclusivos}. Além disso, houve a remoção do item \enquote{Fácil acesso a informação das lojas físicas}, porque devido aos testes executados no primeiro ciclo ficou claro que tal funcionalidade não gerava valor para os usuários.
\item Segmentos de Clientes: A modificação feita foi de \enquote{Compradores que gostam de descontos} para \enquote{Compradores de lojas físicas}. O motivo dessa mudança foi porque o aplicativo não quis focar nos compradores de lojas virtuais, e também dado que o aplicativo também tem a função de ser um panfleto digital, aumentar o escopo para compradores de lojas físicas pode aumentar a base de usuários.
\item Fluxo de Receita: Dada as conversas com os próprios varejistas muitos perguntavam se o usuário teria que pagar alguma quantia para usufruir do aplicativo. A equipe decidiu deixar claro no canvas que o usuário não iria desembolsar nada para utilizar o app.
\end{itemize}

\begin{figure}[H]
\caption{Canvas de Modelo de Negócio após testes}
\centerline{\includegraphics[width=0.8\textwidth]{img/canvas_beeconnect_2}}
\label{fig:canvas_beeconnect_2}
\caption* {Fonte: Elaborado pelo autor}
\caption* {Legenda: As caixas com cores diferentes em relação ao seu bloco são as que sofreram alterações}
\end{figure}

\section{Gerar hipóteses sobre a proposta de valor da empresa 2}
\label{cha:gerar_hipoteses_ 2}
Uma vez que as seis hipóteses elaboradas no ciclo anterior foram respondidas e o canvas do estado atual foi elaborado, o autor se encontrou novamente com o seu orientador para gerar novas hipóteses sobre a proposta de valor. Os dois concordaram que a próxima etapa seria provar o valor do produto para os lojistas e comprovar se os varejistas estavam dispostos a pagar pelo serviço. Então a sétima hipótese elaborada no decorrer do trabalho de conclusão de curso foi:
\begin{itemize}
\item Hipótese 7: \textbf{Os varejistas estão dispostos a pagar pelo serviço}.
\end{itemize}


\section{Desenhar Testes de Hipóteses 2}
\label{cha:desenhar_hipoteses_2}
Após a formulação da hipótese foi feita a estruturação para que tal suposição fosse testada corretamente.

\subsection{Hipótese 7}
\label{cha:hipotese_7}
Para provar que os varejistas estavam dispostos a pagar pelo serviço do aplicativo Beeconnect antes seria necessário provar para os varejistas que o serviço deveras gerava um retorno positivo. Para isso o autor elaborou uma fórmula básica para provar tal hipótese:

\[[(NC * (PC - D)) + (VR * (PC - D))] > PS\]

Onde: 
\begin{itemize}
\item NC: é o numero de novos clientes.
\item PC: é o preço cheio de um produto, ou seja, o valor original de um produto.
\item D: é o valor do desconto aplicado ao produto.
\item VR: é o número de vendas recursivas, isto é o número de compras repetidas feitas por usuários do aplicativo Beeconnect.
\item PS: é o preço do serviço cobrado pela Beeconnect, explicado na \autoref{cha:fluxos_de_receita_canvas_bc_1}.
\end{itemize}

Então seria necessário pelo menos uma loja disposta a fazer uma promoção disponibilizando um cupom exclusivo para testar tal hipótese.
\section{Testar Hipóteses 2}
\label{cha:testar_hipoteses_2}

\subsection{Teste da hipótese 7}
\label{cha:teste_hipotese_7}
A equipe acabou conseguindo três lojas, sendo duas do setor alimentício e uma do setor vestuário. Para o teste com as duas lojas do setor alimentício, uma gelateria e uma sonheria, a equipe organizou um mutirão de marketing, que cobria a região entre as duas lojas, que ficavam localizadas na região dos Jardins entre as ruas Bela Cintra e Augusta, como mostra a \autoref{fig:mutirao_mkt}.
\begin{figure}[H]
\caption{Região coberta pelo mutirão de marketing}
\centerline{\includegraphics[width=0.8\textwidth]{img/mutirao_mkt}}
\label{fig:mutirao_mkt}
\caption* {Fonte: Elaborado pelo autor}
\end{figure}

Durante o mutirão a equipe da Beeconnect distribuiu panfletos para divulgar o aplicativo, distribuiu também alguns chaveiros com a marca da empresa e também carregou alguns balões com gás hélio para chamar mais atenção do publico que passava pela rua. A intenção era que o público baixasse o app na hora e fosse visitar as lojas que estavam disponibilizando descontos exclusivos no aplicativo.

A gelateria disponibilizou a oferta de \textit{\enquote{Gelato médio de R\$15 por R\$11}}. A oferta disponibilizada pela sonheria foi \textit{\enquote{Compre 2 sonhos pague 1}} e por fim a loja de vestuário estava ofertando um sapatênis \textit{\enquote{De R\$299 por R\$199}}.

\section{Analisar Resultados e Repetir Ciclo 2}
\label{cha:analisar_resultados_2}
Os resultados foram medidos ao longo de uma semana após a realização do mutirão de marketing.

\subsection{Resultado do Teste da hipótese 7}
\label{cha:resultado_7}
Após a realização dos testes para confirmar se os varejistas estavam dispostas a pagar pelo serviço, o autor então analisou o resultado de cada uma das lojas.

\subsubsection{Gelateria Casa Elli}
\label{cha:resultado_casa_elli}
O resultado na gelateria Casa Elli foi satisfatório, foram gerados 30 cupons sendo utilizados 26, onde 24 novos clientes que acabaram comprando o gelato médio cujo preço era de R\$15, como mencionado na \autoref{cha:teste_hipotese_7}, mas graças ao desconto pelo aplicativo Beeconnect saiu por R\$11. Os outros dois cupons foram utilizados por usuários antigos do aplicativo, ou seja, se encaixaram nas vendas recursivas como mostra a \autoref{tab:resultado_7_casa_elli_a}.

\begin{table}[H]
\centering
\caption{Resultado do teste 7 na Gelateria Casa Elli}
\label{tab:resultado_7_casa_elli_a}
\begin{tabular}{|c|c|c|c|}
\hline
Novos Clientes & Preço Cheio & Desconto & Vendas Recursivas \\ \hline
24             & 15          & 4        & 2   \\ \hline
\end{tabular}
\caption* {Fonte: Elaborado pelo autor}    
\end{table}

Aplicando os números acima na fórmula mencionada na \autoref{cha:hipotese_7}, surge o resultado abaixo:
\[[(24 * (15 - 4)) + (2 * (15 - 4))] = 286\]

Como a gelateria estava usufruindo da tecnologia de beacons, sua mensalidade custaria R\$99/mês que, quando comparada com resultado de R\$286 advindo do aplicativo Beeconnect em apenas uma semana, pode-se analisar que seria um bom investimento, como mostra a \autoref{tab:resultado_7_casa_elli_b}. 

\begin{table}[H]
\centering
\caption{Análise do teste 7 na Gelateria Casa Elli}
\label{tab:resultado_7_casa_elli_b}
\begin{tabular}{|c|c|c|}
\hline
Resultado & Mensalidade & Retorno do Investimento \\ \hline
286             & 99          &   2.88 \\ \hline
\end{tabular}
\caption* {Fonte: Elaborado pelo autor}    
\end{table}

Apesar do resultado positivo tanto para a gelateria quanto para a Beeconnect, a dona da loja ainda não estava disposta a pagar pelo serviço, ela gostaria de mais tempo para decidir se estava disposta a pagar a quantia. Entretanto, ela disse que utilizaria a plataforma para subir mais campanhas e que daria mais feedbacks para que a equipe conseguisse melhorar ainda mais o produto.

Tal comportamento foi encarado de maneira bastante inesperada pela equipe do aplicativo, que esperava que a varejista já assinasse algum tipo de contrato dizendo que iria pagar pelo serviço disponibilizado pela Beeconnect.

\subsubsection{Sonheria Dulca}
\label{cha:resultado_sonheria_dulca}
O resultado na Sonheria Dulca também foi positivo, foram gerados 21 cupons sendo 19 deles utilizados, onde 18 foram utilizados por novos clientes e 1 foi usufruído por um cliente antigo. Como mencionado anteriormente, a oferta disponiblizada foi de \textit{\enquote{Compre 2 sonhos pague 1}}, na época cada sonho custava R\$10, ou seja, o preço cheio seria de R\$20 e o desconto seria de R\$10, como mostra a \autoref{tab:resultado_7_sonheria_dulca_a}

\begin{table}[H]
\centering
\caption{Resultado do teste 7 na Sonheria Dulca}
\label{tab:resultado_7_sonheria_dulca_a}
\begin{tabular}{|c|c|c|c|}
\hline
Novos Clientes & Preço Cheio & Desconto & Vendas Recursivas \\ \hline
18             & 20          & 10        & 1   \\ \hline
\end{tabular}
\caption* {Fonte: Elaborado pelo autor}    
\end{table}

Aplicando os números acima na fórmula mencionada na \autoref{cha:hipotese_7}, surge o resultado abaixo:
\[[(18 * (20 - 10)) + (1 * (20 - 10))] = 190\]

Assim como a Gelateria, a Sonheria também utilizava a tecnologia de beacons, portanto a mensalidade seria de R\$99/mês. O resultado de R\$190 sem contar com as vendas casadas de água e café foi positivo para a loja, como apresenta a \autoref{tab:resultado_7_sonheria_dulca_b}.

O time da Beeconnect ficou bastante preocupado em virtude da falta de interesse dos varejistas resolverem pagar pelo serviço apesar dos resultados positivos. Seria necessário no próximo ciclo de testes de hipótese algum tipo de mudança para que os varejistas não tivessem como negar o pagamento do serviço.

\begin{table}[H]
\centering
\caption{Análise do teste 7 na Sonheria Dulca}
\label{tab:resultado_7_sonheria_dulca_b}
\begin{tabular}{|c|c|c|}
\hline
Resultado & Mensalidade & Retorno do Investimento \\ \hline
190             & 99          &   1.91 \\ \hline
\end{tabular}
\caption* {Fonte: Elaborado pelo autor}    
\end{table}

Os donos da loja também não se mostraram interessados em pagar pela utilização do serviço e também disseram que não estavam dispostos a utilizar a plataforma por conta própria pois não possuíam uma pessoa disponível para criar as campanhas e administrá-las.

\subsubsection{Tisu Store}
\label{cha:resultado_tisu_store}
No caso da Tisu Store, a oferta do sapatênis com desconto de R\$100 teve 3 cupons gerados, entretanto, não houve utilização desses cupons. Portanto, o resultado foi bem insatisfatório, como mostram a \autoref{tab:resultado_7_tisu_store_a} e a \autoref{tab:resultado_7_tisu_store_b}.


\begin{table}[H]
\centering
\caption{Resultado do teste 7 na Tisu Store}
\label{tab:resultado_7_tisu_store_a}
\begin{tabular}{|c|c|c|c|}
\hline
Novos Clientes & Preço Cheio & Desconto & Vendas Recursivas \\ \hline
0             & 299          & 100        & 0 \\  \hline
\end{tabular}
\caption* {Fonte: Elaborado pelo autor}    
\end{table}

Aplicando os números acima na fórmula mencionada na \autoref{cha:hipotese_7}, surge o resultado abaixo:
\[[(0 * (299 - 100)) + (0 * (299 - 100))] = 0\]

\begin{table}[H]
\centering
\caption{Análise do teste 7 na Tisu Store}
\label{tab:resultado_7_tisu_store_b}
\begin{tabular}{|c|c|c|}
\hline
Resultado & Mensalidade & Retorno do Investimento \\ \hline
0             & 99          &   0 \\ \hline
\end{tabular}
\caption* {Fonte: Elaborado pelo autor}    
\end{table}

Como era de se esperar com esse resultado tão ruim o dono da loja não mostrou nenhum interesse em pagar pelo serviço e também disse que não iria utilizar a plataforma por conta própria. 

\section{Listar Lições Aprendidas do Segundo Ciclo}
\label{cha:listar_licoes_aprendidas_segundo_ciclo}
O segundo ciclo buscou responder se os lojistas iram pagar pelo uso do serviço do aplicativo. Apesar de nenhum parceiro ter se disposto a desembolsar uma quantia mensal para usufruir da plataforma, a equipe tirou aprendizados importantes desse teste.

O primeiro aprendizado foi que o setor de comidas e bebidas apresentou-se como um nicho mais fácil de conseguir novos usuários para o aplicativo. Dado que o valor médio do produto é mais baixo, fica mais fácil o primeiro contato do usuário com o aplicativo, o cliente já vê o benefício muito mais rápido.

A equipe pensou que talvez um outro meio de conseguir que os varejistas paguem talvez seja cobrar também uma taxa a mais pela administração de campanhas no aplicativo. Isso se justifica pelo motivo que muitos não queriam pagar pela mensalidade pelo fato da plataforma ser auto gerida e eles não queriam gastar tempo com gerenciamento da plataforma.

\section{O Fim da Beeconnect}
\label{cha:fim_da_beeconnect}
Infelizmente não foi possível uma execução sucessiva de outros ciclos para testar novas hipóteses sobre o modelo de negócio da Beeconnect. O conselho da holding TM decidiu fechar o projeto com a justificativa que apesar dos aprendizados adquiridos com a Beeconnect, a startup ainda não havia provado o seu modelo de negócio e não parecia haver no horizonte que a situação iria mudar rapidamente.

O sentimento amargo de derrota foi horrível, todo o time ficou desmotivado com a notícia. O grande problema é que não houve um aviso prévio por parte do conselho da holding, a decisão foi repentina, então o time não saiu com um sentimento de dever cumprido, de que tinha tentado com todas as forças todas as alternativas e não estava preparado para tal notícia. Também houve uma sensação no ar que iriam haver demissões.

Apesar das más notícias, o time da startup foi mantido, mas cada um foi suprir uma função diferente dentro da TM. Dois membros foram ajudar no time de ciência de dados na RM. Três membros foram ajudar no time que desenvolve o software que faz mostrar as propagandas da RM nos aplicativos de parceiros. Três membros foram auxiliar no time da BL para construir a plataforma de visualização de análise de dados. Outros dois foram auxiliar no time de vendas da RM para conseguir mais desenvolvedores para a plataforma. O autor ajudou por um tempo na RM integrando com novos parceiros para conseguir disponibilizar as propagandas RM em novos aplicativos, depois o desafio que foi lhe proposto foi de ajudar a desenvolver o banco de dados da BL.

O fim da Beeconnect de fato chocou a todos que estavam fazendo parte dessa jornada, porque toda equipe acreditava bastante no projeto que poderia revolucionar o varejo de lojas físicas. A empresa era composta por pessoas incríveis e competentes, que por falta da utilização de uma metodologia voltada para desenvolvimento de startups acabou sucumbindo a meramente a um status de projeto engavetado.
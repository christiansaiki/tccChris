%!TEX root = index.tex
\chapter[Metodologia]{Metodologia}
\label{chap:metodologia}
Nesse capítulo será apresentado o método utilizado para o desenvolvimento do estudo que visa a tentativa de salvar a empresa Beeconnect. Conforme explicado nos Objetivos do trabalho de formatura, a intenção é provar que empresa possui um modelo de negócio que gera valor para os seus clientes e que seja sustentável. Serão utilizados os conceitos apresentados na revisão bibliográfica, principalmente os textos elaborados por \citeonline{leanstartup}, \citeonline{startupowners} e \citeonline{businessmodel}.

A metodologia utilizada nesse trabalho de conclusão de curso foi dividida nos seguintes tópicos:
\begin{enumerate}
\item Mapear estado atual da startup
\item Gerar hipóteses sobre a proposta de valor da empresa.
\item Desenhar os testes de hipóteses.
\item Testar hipóteses.
\item Analisar resultados e repetir o ciclo.
\item Listar lições aprendidas.
\end{enumerate}

\section{Mapear estado atual da startup}
\label{cha:mapear_estado}
Para mapear o estado atual da startup baseado no aplicativo Beeconnect foi utilizado o conceito do Canvas de Modelo de Negócio apresentado por \citeonline{businessmodel}.
O autor com a ajuda dos demais membros da Beeconnect deverá elaborar no escritório da empresa um canvas de modelo de negócio que reflita o estado atual da empresa. 

\section{Gerar hipóteses sobre a proposta de valor da empresa}
\label{cha:gerar_hipoteses}
Para gerar as hipóteses sobre a proposta de valor da empresa foram utilizados os conceitos de Validação do Cliente de \citeonline{startupowners}, o capítulo de Experimentação de Startups de \citeonline{leanstartup}, e os conceitos propostos por \citeonline{businessmodel} e \citeonline{valueproposition}.    

Uma vez que o Canvas de Modelo de Negócio da Beeconnect estiver pronto o autor deverá se reunir com o professor André Fleury na Escola Politécnica para gerar as hipóteses sobre a proposta de valor da empresa. Tais suposições deverão conectar os blocos Segmentos de Clientes e Proposição de Valor.

\section{Desenhar os testes de hipóteses}
\label{cha:desenhar_hipoteses}
Após gerar as hipóteses o autor também deverá recorrer a mentoria do professor André Fleury para que os testes realmente testem as hipóteses. Juntos os dois deverão desenhar os testes e as métricas que definirão se a Beeconnect passou ou não no teste. 
Além disso o autor deverá se reunir com sua equipe para planejar como os testes serão implementados. A equipe terá que responder as perguntas abaixo para cada teste:
\begin{itemize}
\item Quando o teste será realizado?
\item Quem realizará o teste?
\item Onde o teste será executado?
\end{itemize}

\section{Testar hipóteses}
\label{cha:testar_hipoteses}
Após planejar como realizar cada teste com o intuito de testar as hipóteses de valor, a equipe da Beeconnect deverá ir a campo responder cada uma das suposições geradas.

\section{Analisar resultados e repetir ciclo}
\label{cha:analisar_resultados}
Uma vez que os testes forem realizados e os resultados obtidos o autor deverá checar se a Beeconnect passou ou não nos testes, ou seja, se as hipóteses foram provadas ou se foram refutadas. 
Após tal análise o autor deverá se reunir novamente com o professor André Fleury para que juntos eles iterem novamente pelo ciclo de modo que mais hipóteses sejam provadas ou refutadas e que o modelo de negócio da Beeconnect fique mais claro.

\section{Listar lições aprendidas}
\label{cha:listar_licoes_aprendidas}
De modo que o todo o aprendizado dos estudos e testes realizados no decorrer desse trabalho de conclusão de curso não fique perdido, será realizado um momento de reflexão para que o autor liste quais foram as lições aprendidas no decorrer dessa jornada. O autor deverá apontar os erros cometidos por ele e sua equipe durante o desenvolvimento da Beeconnect, bem como deverá apontar como a literatura recomenda para que outros empreendedores não cometam o mesmo erro.
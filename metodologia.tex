%!TEX root = index.tex
\chapter[Metodologia]{Metodologia}
\label{chap:metodologia}
Nesse capítulo será apresentado o método utilizado para o desenvolvimento do estudo que visa a tentativa de salvar a empresa Beeconnect. De acordo com a introdução realizada no Capítulo 1, a Beeconnect estava prestes a ser encerrada pelo Conselho da holding Techmob. Entre os motivos estavam: o fato da empresa estar gastando muito em recursos humanos, quase quinhentos mil reais no decorrer de um ano, e o fato da startup ainda não ter provado o seu modelo de negócio. Mesmo após meses e meses de trabalho árduo nenhum varejista ainda estava disposto a pagar pelo serviço oferecido pelo aplicativo.

Conforme explicado nos Objetivos do trabalho de formatura, a intenção é provar que empresa possui um modelo de negócio que gera valor para os seus clientes e que seja sustentável, ou seja, os varejistas devem estar dispostos a pagar pelo serviço e ao mesmo tempo deve haver sobrar uma margem de lucro positiva. Foram utilizados os conceitos apresentados na revisão bibliográfica, principalmente os textos elaborados por \citeonline{leanstartup}, \citeonline{startupowners} e \citeonline{businessmodel}.

A metodologia utilizada nesse trabalho de conclusão de curso foi dividida nos seguintes tópicos e ilustrada na \autoref{fig:metodologia}:
\begin{enumerate}
\item Mapear estado atual da startup
\item Gerar hipóteses sobre a proposta de valor da empresa.
\item Desenhar os testes de hipóteses.
\item Testar hipóteses.
\item Analisar resultados e repetir o ciclo.
\item Listar lições aprendidas.
\end{enumerate}

\begin{figure}[H]
\caption{Metodologia utilizada}
\centerline{\includegraphics[scale=0.25]{img/metodologia}}
\label{fig:metodologia}
\caption* {Fonte: Elaborado pelo autor}
\end{figure}

\section{Mapear estado atual da startup}
\label{cha:mapear_estado}
Para mapear o estado atual da startup baseado no aplicativo Beeconnect foi utilizado o conceito do Canvas de Modelo de Negócio apresentado por \citeonline{businessmodel}.
O autor contou com a ajuda dos demais membros da Beeconnect para elaborar no escritório da empresa um canvas de modelo de negócio, preenchendo cada um dos nove blocos de modo a permitir uma melhor visualização e compreensão do estado atual da empresa na época. 

Conforme recomendado por  \citeonline{businessmodel}, foram utilizados \textit{post-its} que ficaram grudados em uma lousa o que facilitou muito as alterações, assim foram feitas diversas iterações até que finalmente chegou-se em um modelo que agradou a todos. Depois, o autor passou o modelo para o \textit{Google Docs} para que todos tivessem fácil acesso ao Canvas, e que também permitiu o compartilhamento de tal modelo com o orientador do trabalho de conclusão de curso.

\section{Gerar hipóteses sobre a proposta de valor da empresa}
\label{cha:gerar_hipoteses}
Para gerar as hipóteses sobre a proposta de valor da empresa foram utilizados os conceitos de Validação do Cliente de \citeonline{startupowners}, o capítulo de Experimentação de Startups de \citeonline{leanstartup}, e os conceitos propostos por \citeonline{businessmodel} e \citeonline{valueproposition}.    

Uma vez que o Canvas de Modelo de Negócio da Beeconnect estivesse pronto o autor reuniu-se com o seu orientador na Escola Politécnica para gerar as hipóteses sobre a proposta de valor da empresa. Tais suposições conectaram os blocos Segmentos de Clientes e Proposição de Valor de tal forma que fosse possível verificar se os clientes estavam realmente enxergando valor nas propostas que o aplicativo Beeconnect estava oferecendo. Ou seja, através dos testes de tais hipóteses seria possível entender se o aplicativo conseguiria atender as necessidades de seus dois segmentos de clientes distintos, os usuários do aplicativo e os varejistas.

\section{Desenhar os testes de hipóteses}
\label{cha:desenhar_hipoteses}
Após gerar as hipóteses o autor também recorreu a mentoria do orientador para que os testes realmente testassem as hipóteses. Juntos os dois desenharam os testes e as métricas que definiriam se a Beeconnect passou ou não no teste. 
Além disso o autor reuniu-se com sua equipe para planejar como os testes seriam implementados. A equipe respondeu as perguntas abaixo para cada teste:
\begin{itemize}
\item Quando o teste será realizado?
\item Quem realizará o teste?
\item Onde o teste será executado?
\item Quanto custaria para realizar o teste?
\item Quanto tempo levaria para realizá-lo?
\end{itemize}

\section{Testar hipóteses}
\label{cha:testar_hipoteses}
Após planejar como realizar cada teste com o intuito de testar as hipóteses de valor, a equipe da Beeconnect foi a campo responder cada uma das suposições geradas. Basicamente, foi a aplicação do conceito do \textit{Genchi Genbutsu} de "saia e veja por si mesmo" utilizado na manufatura enxuta e mencionado por \citeonline{leanstartup}.

Os membros que foram a campo deveriam levar um caderno para que anotassem os aprendizados para depois compartilhar com os demais integrantes da empresa, de modo a preencher o máximo possível dos pontos destacados na \autoref{fig:intrucoes_mutirao}. 

\begin{figure}[H]
\caption{Instruções do mutirão de vendas}
\centerline{\includegraphics[width=1.0\textwidth]{img/intrucoes_mutirao}}
\label{fig:intrucoes_mutirao}
\caption* {Fonte: Manual de vendas da Beeconnect}
\end{figure}

Também foi recomendado aos participantes do mutirão registrar outras informações como as reações dos varejistas quando expostos a diferentes argumentos de venda como "Tecnologia de beacons" ou "Notificações no aplicativo" além de quais dúvidas que os usuários e varejistas poderiam ter a respeito do aplicativo.

\section{Analisar resultados e repetir ciclo}
\label{cha:analisar_resultados}
Uma vez que os testes foram realizados e os resultados obtidos o autor checou se a Beeconnect passou ou não nos testes, ou seja, se as hipóteses foram provadas ou se foram refutadas. 
Após tal análise o autor reuniu-se novamente com o seu orientador para que juntos eles iterassem novamente pelo ciclo de modo que mais hipóteses fossem provadas ou refutadas e que o modelo de negócio da Beeconnect ficasse cada vez mais claro. Foi feito o maior número possível de iterações do ciclo até que todas as hipóteses fossem testadas ou até que o Conselho da holding Techmob decidisse encerrar o projeto. A ideia de repetir o ciclo foi baseada no conceito do Ciclo de Feedback Construir-Medir-Aprender de \citeonline{leanstartup}.

\section{Listar lições aprendidas}
\label{cha:listar_licoes_aprendidas}
De modo que o todo o aprendizado dos estudos e testes realizados no decorrer desse trabalho de conclusão de curso não ficassem perdidos, foi realizado um momento de reflexão para que o autor listasse quais foram as lições aprendidas no decorrer do final de cada iteração do processo de validação de hipóteses. Tais aprendizados deveriam ser expostos no \textit{Google Drive} da empresa para facilitar o acesso e visualização.

O objetivo de listar as lições aprendidas era que as pessoas da holding Techmob não cometessem os mesmo erros cometidos pela equipe da Beeconnect além de poupar um tempo valioso que no mundo do empreendedorismo pode fazer total diferença entre o sucesso e falha no negócio. Entre as lições poderiam conter quais as melhores estratégias de marketing até coisas banais como não esquecer de levar uma câmera digital profissional e um dispositivo de armazenamento digital ao visitar um cliente lojista.
%!TEX root = index.tex
\chapter[Revisão Bibliográfica]{Revisão Bibliográfica}
\label{chap:revisao}

Nesse Capítulo serão apresentados os principais conceitos usados no desenvolvimento do método e na resolução do problema apresentado.

O Capítulo será dividido em dois grandes blocos: Startup Enxuta e Estratégia.

O bloco Startup Enxuta irá focar nos conceitos de administração de startups. Serão apresentadas as propostas de Steven Blank e Bob Dorf sobre o Desenvolvimento de Clientes. Além disso serão abordadas as propostas de Eric Ries sobre elaboração de hipóteses e testes para buscar a Aprendizagem Validada.

Já o bloco Estratégia focará no conceito de Canvas de Modelo de Negócio, Canvas de Proposição de Valor e em alguns pontos importantes de Peter Thiel.

\section{Startup Enxuta}
\label{cha:startupenxuta}
As revisões bibliográficas contidas nesse bloco irão focar nos principais conceitos da Startup Enxuta. \citeonline{leanstartup} define uma startup da seguinte maneira: "Uma startup é uma instituição humana projetada para criar novos produtos e serviços sob condições de extrema incerteza." (RIES, 2011, p. 24). \citeonline{leanstartup} afirma que a Startup Enxuta é um conjunto de práticas para aumentar as possibilidades de sucesso de uma startup. Tais práticas foram baseadas no sistema de manufatura enxuta que nasceu no Japão com o Sistema de Produção Toyota idealizado por Taiichi Ohno e Shigeo Shingo.

\subsection{O método da Startup Enxuta}
\label{cha:metodostartupenxuta}
As startups são empresas inovadoras, disruptivas e caóticas e apesar disso elas necessitam de gestão, afirma \citeonline{leanstartup}. Muitas startups fracassam por tentarem aplicar metodologias antigas de administração que focam em um bom planejamento, estratégia sólida e uma pesquisa de mercado completa. Entretanto, segundo \citeonline{leanstartup}, planejamento e previsão só são precisos quando se tem um histórico operacional longo e estável oque não se aplica no meio no qual as startups estão imersas. Outras startups fracassam por adotarem a prática de "simplesmente faça" e repudiarem qualquer tipo de gestão.

O método da Startup Enxuta é divido em três partes: "Visão", "Direção" e "Aceleração". Entretanto, serão abordados só as duas primeiras nessa revisão bibliográfica dada que a terceira parte se aplica a startups que já possuem grande parte de suas hipóteses validadas.

Na parte "Visão", \citeonline{leanstartup} introduz o conceito da aprendizagem validada que é uma maneira de medir se as startups estão progredindo. Na parte "Direção", encontram-se os importantes conceitos do ciclo básico de feedback Contruir-Medir-Aprender e da contabilidade para inovação.


\subsubsection{Aprendizado Validado}
\label{cha:apredizado_validado}
\citeonline{leanstartup} afirma que uma startup não existe apenas para fabricar coisas e ganhar dinheiro. A função dela é aprender a desenvolver um negócio sustentável e que tal aprendizagem pode ser validada cientificamente por meio de experimentos frequentes. E assim como na manufatura enxuta, o aprendizado de onde e quando investir recursos resulta em economia de tempo e dinheiro.

O objetivo de uma startup é descobrir a coisa certa a criar, a coisa que os clientes querem e pela qual pagarão, o mais rápido possível. Segundo \citeonline{leanstartup}, aprendizagem validada é o processo de demonstrar empiricamente que uma equipe descobriu verdades valiosas acerca das perspectivas de negócios presentes e futuras de uma startup.

O aprendizado validado é respaldado por dados de clientes reais pois implica na melhora dos indicadores chave da empresa nascente. Para a startup ser produtiva ela deve buscar sistematicamente as coisas certas para serem desenvolvidas. Ao invés de gastar tempo pensando se o produto pode ser desenvolvido, a empresa deve se perguntar se ela deve desenvolver tal produto ou tal serviço, e além disso, deve se perguntar se é possível desenvolver um negócio sustentável em torno desses produtos e serviços. \cite{leanstartup}

Deste modo, \citeonline{leanstartup}, recomenda que cada funcionalidade, produto e campanha publicitária sejam tratados como se fossem um experimento científico para alcançar a aprendizagem validada.

Com a aprendizagem validada sendo executada com sucesso a startup escapa de um grande problema: gastar tempo e recursos preciosos para construir algo que ninguém quer.

\subsubsection{Experimentação da Startup}
\label{cha:experimentacao_da_startup}

\citeonline{leanstartup} afirma que uma das lições mais importantes do método científico é que para aprender você deve poder fracassar. O objetivo de todo experimento associado à empresa nascente é descobrir como desenvolver um negócio sustentável em torno da visão \cite{leanstartup}.

O modelo da startup enxuta confere um método rápido para testar as hipóteses que permeiam a visão da empresa de modo a mitigar desperdícios ao longo do caminho.

Tal método tem como primeiro passo dividir a visão em partes menores. Segundo \citeonline{leanstartup}, as hipóteses de valor e de crescimento são as duas suposições mais importantes para os empreendedores.

A hipótese de valor tem como objetivo testar se o produto ou serviço deveras fornece valor para os clientes no momento em que estão o usufruindo. Por outro lado, a hipótese de crescimento tem em sua concepção testar como os novos clientes irão achar um produto ou serviço. \cite{leanstartup}.

\citeonline{leanstartup}, afirma que o experimento deve ser encarado com um primeiro produto ao invés de uma simples pesquisa teórica. Tais experimentos/produtos irão variar em sua complexidade de modo a garantir a aprendizagem validada.

\subsubsection{Ciclo de Feedback Construir-Medir-Aprender}
\label{cha:ciclo_cma}

O ciclo de feedback construir-Medir-Aprender está no centro da startup enxuta. Segundo \citeonline{leanstartup}, a atividade fundamental de uma startup é transformar ideias em produtos e/ou serviços, medir como os clientes reagem às iterações deles e depois aprender se é necessária a pivotação ou se é possível perseverar na mesma direção. Todos os processos da startup devem ter como norte percorrer esse ciclo o mais rápido possível.

O planejamento de como percorrer o ciclo Construir-Medir-Aprender deve ser feito na ordem inversa. Primeiro, deve-se pensar qual aprendizado o experimento está buscando. Posteriormente deve-se planejar quais medições serão realizadas a fim de se determinar se houve de fato uma aprendizagem validada. Só no final deve-se pensar qual produto será desenvolvido com a finalidade de executar o experimento e obter tais medições. \cite{leanstartup}

\begin{figure}[H]
\caption{Ciclo de Feedback Construir-Medir-Aprender}
\centerline{\includegraphics[scale=0.5]{img/ciclo_construir_medir_aprender}}
\label{fig:ciclo_construir_medir_aprender}
\caption* {Fonte: \citeonline{leanstartup}}
\end{figure}

\subsubsection{Atos de fé}
\label{cha:atos_de_fe}

\citeonline{leanstartup} afirma que o papel da estratégia nas startups é descobrir as perguntas certas para se fazer. O primeiro desafio de um empreendedor é transformar a sua empresa em uma máquina de testes para responder tais perguntas sistematicamente. O segundo desafio é continuar conduzindo testes rigorosamente sem perder de vista a visão geral da empresa.

Todo plano de negócios começa com uma série de hipóteses. O plano traça uma estratégia que considera essas hipóteses verdadeiras e prossegue mostrando como alcançar a visão da empresa. \citeonline{leanstartup} chama as suposições mais importantes de atos de fé, porque todo o sucesso do empreendimento depende da veracidade de tais hipóteses. Se elas se provarem verdadeiras, um grande sucesso acontecerá. Caso contrário a startup pode estar fadada ao fracasso.

\citeonline{leanstartup} ressalta a importância do conceito \textit {Genchi Gembutsu} (vá e veja por si mesmo) da manufatura enxuta. Tal expressão japonesa implica na relevância de basear decisões estratégicas na compreensão direta de seus clientes. Os dados dos clientes que devem ser recolhidos só existem fora do escritório.

\subsubsection{Testes}
\label{cha:testes}

Na parte Construir do ciclo Construir-Medir-Aprender o método da startup enxuta prega o emprego do conceito do MVP que é a sigla em inglês para Minimum Viable Product, cujo significado é Produto Mínimo Viável.

Segundo \citeonline{leanstartup}, o MVP tem como finalidade ajudar os empreendedores a começar o processo de aprendizagem o mais rápido possível tendo o objetivo de testar hipóteses fundamentais do negócio. Ele é apenas o primeiro passo na grande jornada em busca do aprendizado validado, após um certo número de iterações o empreendedor pode descobrir que sua estratégia era falha e pode decidir mudar de rumo para tentar reconquistar sua visão.

\begin{figure}[H]
\caption{Visão da Startup}
\centerline{\includegraphics[scale=0.5]{img/pivotar}}
\label{fig:pivotar}
\caption* {Fonte: \citeonline{leanstartup}}
\end{figure}

A lição do MVP é que qualquer trabalho adicional além do requerido para iniciar o ciclo de aprendizagem é considerado desperdício. O empreendedor deve se lembrar

que mesmo um MVP de baixa qualidade pode estar atuando em prol de um grande produto de qualidade de ponta. E que enquanto não souber quem é o cliente, o empreendedor jamais saberá o que é qualidade. \cite{leanstartup}

\citeonline{leanstartup}, também cita os conceitos dos adotantes iniciais e de feedback constante. Os adotantes iniciais devem ser os primeiros clientes, pois eles se importam mais com o fato de serem os primeiros a utilizar um produto novo e são mais indulgentes com falhas de projeto do que os clientes normais. O feedback constante baseia-se em sempre ter clientes para testar protótipos para a melhoria contínua em cima das iterações do produto.

\subsubsection{Medir}
\label{cha:medir}

Em seus primeiros momentos a startup apenas admira os belos números estimados no plano de negócios. \citeonline{leanstartup} propõe duas tarefas para a startup. A primeira é medir rigorosamente onde ela está naquele momento (baseline). A segunda é iterar através de experiências para mover os números para cima buscando atingir a meta idealizada no plano de negócios.

Muitas empresas enfrentam dificuldades para saber se as mudanças feitas nos produtos trouxeram algum resultado tanto para melhor ou pior. Outra dificuldade é saber se foram extraídas as lições corretas dessas mudanças. \citeonline{leanstartup} então propõe o modelo da contabilidade para inovação, sua principal finalidade é tornar os saltos de fé em modelos financeiros quantificados. O modelo funciona em três etapas:
\begin{enumerate}
\item Utilizar um MVP para adquirir dados reais e determinar onde a empresa se encontra no momento.
\item Iterar através de experimentos para tentar melhorar os números de forma a alcançar o plano.
\item Depois de um certo número de iterações de forma a melhorar o produto a startup deve decidir se é hora de pivotar, mudar a estratégia, ou se deve insistir na mesma estratégia.
\end{enumerate}

Segundo \citeonline{leanstartup}, a análise de coorte é uma das ferramentas mais importantes ao analisar uma startup. Ao invés de olhar para números acumulados ou números brutos como receita total e número total de usuários, o autor propõe que a startup considere o desempenho de cada grupo de clientes que entra em contato com o produto independentemente. Cada grupo é chamado de coorte.

A análise de coorte é útil em diversos tipos de negócio, pois cada empresa depende, para sua sobrevivência, de sequências de comportamentos de clientes denominadas fluxos. Os fluxos de clientes regem a interação dos clientes com os produtos de uma empresa, permitem compreender um negócio em termos quantitativos e apresentam m poder preditivo muito maior do que a métrica bruta tradicional. \cite{leanstartup}

\subsection{Desenvolvimento de Clientes}
\label{cha:desenvolvimento_de_clientes}
Muitas startups tem uma ideia, propõem uma solução e definem seu modelo de negócios baseado em suposições que elas tem do mercado. Mas nem sempre as suas suposições são as mais adequadas, e dessa forma elas somente descobrem isso quando o produto é lançado no mercado.

Segundo \citeonline{startupowners}, o modelo de desenvolvimento de produtos tem sido o grande responsável pelo fracasso de muitas startups. Muitas startups da época da bolha das ponto-com tinham como  em comum que o seu maior componente de risco não era tecnólogico e sim de mercado. \citeonline{startupowners} explica os problemas que há neste modelo. Diversas são as justificativas que \citeonline{startupowners} usa para explicar esses problemas. Uma delas é o fato de a grande maioria dos empreendedores serem adeptos de uma cultura baseada em opiniões e não em fatos. Ou seja, as hipóteses são estabelecidas mas quase não existe a preocupação em validá-las. Outra justificativa que Blank cita é o fato de grande parte das startups focarem na execução e não no aprendizado. O empreendendor mesmo tendo conhecimento de que existem várias hipóteses não testadas, utiliza o plano de negócios para acompanhar a execução

A metodologia de desenolvimento de clientes proposta por \citeonline{startupowners} considera que tudo definido no problema e na solução são apenas hipóteses que precisam ser testadas, validadas e renovadas. Dessa forma, depois de uma validação no mercado, é possível que a ideia inicial seja completamente modificada. A metodologia propõe um processo iterativo e paralelo ao desenvolvimento do produto, criando um plano de negócios cada vez mais voltado à necessidade real do cliente e diminuindo assim o risco do produto não ser aceito no mercado.

\citeonline{startupowners} propôs um modelo de desenvolvimento de clientes que é dividido em quatro etapas que serão detalhadas abaixo:
\begin{enumerate}
\item Descoberta do cliente.
\item Validação do cliente.
\item Criação do cliente.
\item Construção da Empresa
\end{enumerate}

\begin{figure}[H]
\caption{Processo de desenvolvimento de clientes}
\centerline{\includegraphics[scale=0.5]{img/desenvolvimento_de_clientes}}
\label{fig:desenvolvimento_de_clientes}
\caption* {Fonte: \citeonline{startupowners}}
\end{figure}

\subsubsection{Descoberta do cliente}
\label{cha:descoberta_do_cliente}
Nesta etapa do processo, \citeonline{startupowners} descrevem como a startup deve encontrar o alinhamento entre o problema e a solução.
\citeonline{startupowners} defendem a premissa de parar de vender, começar a ouvir. Dentro da organização há apenas opiniões, não fatos. Para encontrar os fatos, o empreendedor deve buscá-los fora.
Outro item que \citeonline{startupowners} defende é o teste de hipóteses. Duas hipóteses fundamentais que devem ser testadas são: 
\begin{itemize}
\item Concepção do problema
\item Concepção do produto
\end{itemize}

Segundo \citeonline{startupowners}, o empreendedor deve buscar quais são os maiores problemas do cliente e verificar se o produto de fato resolve esses problemas. Saber quanto o cliente pagará para resolver esses problemas e se eles concordam com a solução é fundamental. É nesta fase que a organização deve gastar boa parte do tempo refletindo sobre questões como o público alvo do produto, as necessidades desse público, qual o provável modelo de negócio (custos, preço do projeto, recursos chave, etc), os concorrentes do produto, as características do produto, o tipo de mercado, benefícios do produto para os usuários, entre outros. Essa fase é um ciclo, que só deve ser terminado quando se tiver a certeza que o produto realmente resolve o problema de alguém.

\begin{figure}[H]
\caption{Descoberta do Cliente}
\centerline{\includegraphics[scale=0.5]{img/descoberta_do_cliente}}
\label{fig:descoberta_do_cliente}
\caption* {Fonte: \citeonline{startupowners}}
\end{figure}

\citeonline{startupowners} afirmam que para \textit{websites} e para aplicativos a descoberta do cliente começa quando a primeira versão desse \textit{website} ou aplicativo está no ar. Assim os empreendedores já conseguem testar suas hipóteses baseados nesse produto mínimo viável e ajustar as estratégias de aquisição de clientes iterativamente. Tal tática foi utilizada por empresas como Facebook e Groupon que começaram a jornada por busca de clients com produtos mal acabados.

\subsubsection{Validação do cliente}
\label{cha:validacao_do_cliente}
O processo de validação do cliente, descrito por \citeonline{startupowners}, é o segundo passo do processo de desenvolvimento de clientes e oferece aos empresários uma forma de desenvolver os conhecimentos necessários para projetar seu modelo de negócio. 

Nesta fase, \citeonline{startupowners} afirma que a seguinte questão deve ser respondida: Os clientes pagarão pelo seu produto? O empreendedor deve entender como funciona o ciclo de vendas e o modelo financeiro da empresa. O processo de vendas e de distribuição do produto devem ser validados. O objetivo aqui é encontrar um modelo de vendas adaptável e escalável e validá-lo. 

Segundo \citeonline{startupowners}, a Validação do Cliente é um método que permite o desenvolvimento de um processo de vendas. Ao final da etapa de validação deve provar que o empreendedor encontrou um mercado e um conjunto de clientes que reagem positivamente ao produto. É uma fase que valida o processo de vendas e marketing, onde tudo pode mudar, inclusive o cliente. Caso isso venha a acontecer, deve-se voltar para a etapa de Descoberta do Cliente onde um novo tipo de cliente será análisado. Tal retorno é conhecido como Pivô. Essa fase também é um ciclo de adaptação e melhoramento do plano de negócios.

\begin{figure}[H]
\caption{Validação do Cliente}
\centerline{\includegraphics[scale=0.5]{img/validacao_do_cliente}}
\label{fig:validacao_do_cliente}
\caption* {Fonte: \citeonline{startupowners}}
\end{figure}

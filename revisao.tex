%!TEX root = index.tex
\chapter[Revisão Bibliográfica]{Revisão Bibliográfica}
\label{chap:revisao}

\section{Parceria Universidade-Empresa}
\label{cha:ensino}

Ary Plonski:

"Nos últimos anos, a ênfase da discussão sobre a cooperação entre a academia e o chamado setor produtivo tem-se deslocado da temática ideológica para a da gestão"
"Do lado empresarial, hoje são raras as manifestações, ao menos públicas, de confronto ideológico com a instituição universitária."

"(Dado o alinhamento ideológico entre as partes), um fator crítico para o êxito da cooperação é a gestão adequada da interface em seus vários níveis - desde o alinhamento de percepções dos cooperantes a respeito de quais são os diferentes objetivos colimados com a relação e os condicionantes que cada cultura impõe, até a administração cotidiana dos projetos e atividades envolvidas na transformação dos objetivos estipulados em resultados tangíveis."

"Os artigos apresentados nesta edição da RAUSP permitem identificar e caracterizar alguns dos mais importantes desafios gerenciais para tornar a cooperação empresa-universidade não apenas mutuamente benéfica, mas também uma relação transformadora. Dentre eles, destacam-se os seguintes: 
- compartilhar uma visão multidimensional e integrada da cooperação universidade-empresa, centrada no desenvolvimento de competências humanas

.: A parceria entre a universidade e o setor produtivo dá-se primeiramente no plano de ensino de graduação, com o aproveitamento de quadros profissionais formados pela academia em escalões superiores das empresas. (Gabao - Isso significa que o desenvolvimento de competências para alunos de graduação refletirão em bons profissionais para as empresas, e esse tipo de parceria não envolve algo garantido entre empresa e universidade, mas é um dos exemplos de atuação cooperativa entre ambas as partes, por isso deve-se existir essa visão multidimensional e integrada)

- perceber com clareza as missões distintas, mas complementares, da empresa e da universidade no processo de inovação

.: De um lado estão as universidades empreendedoras (\textit{entrepreneurial universities}). Apresentadas em foros internacionais como um modelo inspirador para a universidade no próximo século	, enfatizam, a par de excelência acadêmica tradicional, um papel ativo da universidade no mercado do conhecimento. De outro lado estão as universidades corporativas (\textit{corporate universities}), entidades criadas por empresas para "formar e desenvolver os talentos humanos na gestão dos negócios", conforme definição apresentada por Marisa Eboli em seu artigo.  Cabe observar, preliminarmente, que não é papel usual da universidade - e sim da empresa - desenvolver tecnologia, entendida como conhecimento organizado aplicável à produção de bens e serviços.

- desenvolver respostas inovativas às diversas necessidades de cooperação empresa-universidade

.: Merecem destaque dois aspectos relevantes dessa experiência para a gestão da cooperação, no contexto da sociedade do conhecimento. O primeiro é que, mesmo em uma relação institucional assimétrica, a verdadeira cooperação envolve aprendizado por ambas as instituições, pois "a Universidade beneficia-se com a compreensão das reais necessidades da sociedade e as empresas passam a ter acesso ao enorme acervo de conhecimentos, sendo beneficiadas com soluções rápidas, além de poderem capacitar-se melhor". 

- capacitar para a gestão eficaz da cooperação empresa-universidade"

.: A gestão adequada da cooperação entre a academia e o segmento produtivo requer conhecimentos, habilidades e atitudes apropriadas para lidar com questões estratégicas - começando pela missão e pela visão institucional - táticas, como a da propriedade intelectual e a do equacionamento econômico-financeiro mais favorável, e operacionais, como a gestão de projetos, frequentemente pluri-institucionais, capazes de transformar desejos em resultados. 

\begin{itemize}
\item Clark Burton : Pursuing the Entrepreneurial university
\end{itemize}

\section{Metodologia de Entrevistas}
\label{cha:ensino}

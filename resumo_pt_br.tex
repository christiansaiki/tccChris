%!TEX root = index.tex
% resumo em português
\setlength{\absparsep}{18pt} % ajusta o espaçamento dos parágrafos do resumo
\begin{resumo}
Esse trabalho de conclusão de curso de Engenharia de Produção da Escola Politécnica da Universidade de São Paulo teve como objetivo reestruturar uma startup que teve um nascimento turbulento. Foi utilizado um método que teve como base a utilização dos princípios da Startup Enxuta, do Desenvolvimento de Clientes e do Canvas de Modelo de Negócio.

A empresa nascente desse trabalho desenvolveu um aplicativo de descontos em lojas físicas, cujo diferencial era a utilização da tecnologia de beacons para conseguir uma melhor precisão para geolocalização em ambientes internos. Assim, o aplicativo conseguiria oferecer o melhor desconto para usuário baseado na localização do mesmo. O principal desafio da empresa era crescer tanto a base de usuários quanto o número de lojas presentes no aplicativo.

O autor desse trabalho é um dos fundadores da empresa e esteve presente em todo os processos decisivos e no desenvolvimento do aplicativo. Para que o trabalho fosse desenvolvido o autor elaborou um método para que fosse possível testar as hipóteses essenciais da empresa de forma sistemática. Além disso, o método demandava que o aprendizado adquirido durante cada ciclo de testes fosse registrado para que a empresa não cometesse os mesmos erros novamente.

No primeiro ciclo de testes o objetivo foi testar se os usuários viam valor no aplicativo, se era possível conseguir novos usuários de forma barata utilizando as redes sociais e se os varejistas tinham interesse em aderir a esse novo aplicativo.

No segundo ciclo de testes o objetivo foi checar se os varejistas estavam dispostos a pagar pela utilização do serviço, dado que até o momento nenhum parceiro estava pagando pelo uso da plataforma disponibilizada pelo aplicativo, ou seja, a receita da empresa era nula.

O trabalho apresentado é o resultado de uma busca para tentar salvar uma empresa, cujo sonho era o de revolucionar o varejo físico brasileiro e um dia o do mundo.

\textbf{Palavras-chaves}: Empresa, Nascente, Enxuta, Beacons, Varejo.
\end{resumo}